%************************************************
\chapter{Introduction}\label{ch:introduction}
%************************************************
\glsresetall % Resets all acronyms to not used



TCP in wireless multi-hop networks (WMNs) generally comes with bad performance, especially if there are multiple crossflows in the topology \cite{wcp}. This problem occurs because of incorrect assumptions of congestion from the congestion control. Whereas congestion control in general would prevent congestion in the network in wired topologies satisfactorily, it does not in WMNs, in particular because the assumption that congestion is indicated when a packet is lost is not sufficient in WMNs. In wireless networks in general, packet loss is not a good indication for congestion, because packet loss can be caused by interferences or general transmission errors, which are not happening because of congestion. So conventional congestion control would react incorrectly and thus unnecessarily reduce throughput. \\
Conventional TCP is also not considering fairness. With multiple flows in the network, congestion control can also take care that no flow is overpowering others, thus congesting the network. Conventional TCP is prone to be unfair, as it does not explicitly take fairness into account. Isolated flows are prone to perform better than flows that are interfering with each other because these have to be coordinated or some are overpowering the rest to an extend that the overpowered flows can not send any data. \\
Most algorithms are human-made and are made to just cope with the corresponding problem. In the case of congestion control in WMNs conventional congestion control gets in contact with problems that were not considered by their designers. Also these congestion control algorithms were built without any other optimisation aim than dealing with congestion. So in WMNs, conventional congestion control would cut flows for the sake of congestion prevention.\\
As opposed to the well-known conventional congestion control algorithms, recent research has demonstrated innovative approaches to automatically generate TCP congestion control algorithms in a data-driven fashion.
Remy \cite{remy} is a prominent example of such algorithms that optimises in the direction of a global target function for a given network. It can optimise in relation to, for example a combination of throughput and delay and achieve satisfactory measures while preventing congestion. The generated congestion control algorithm is decentralised and works only with locally available features.\\
iTCP \cite{iTCP} is another innovative learning-based approach to congestion control which is a reinforcement-learning based neural network, that was created for wireless ad-hoc networks. Neural networks are a good choice for optimisation problems, since they can implement even complex behaviour, while being small in size.\\
We combine the idea of data-driven optimisation of Remy and the idea of teaching a neural network to perform congestion control and present NEAT-TCP. We use NEAT (NeuroEvolution of Augmenting Topologies) \cite{neat} as our learning method. It evolves neural networks and is used for optimisation and control problems.

\section{Contributions}\label{sec:contribution}
There are four main contributions of this work:
\begin{itemize}
	\item NEAT-TCP: Design, implementation and evaluation of a technique to generate neural networks for congestion control by neuroevolution. 
	\item Porting of iTCP, a recently published neural network based congestion control algorithm for WMNs, which serves as a performance benchmark in this work, from ns-2 to ns-3.
	\item Generation of two different flavours of congestion control algorithms with NEAT-TCP: 
	\begin{itemize}
	\item NEAT-TCP1 optimises a combination of both the overall throughput and the fairness according to Jain`s 			fairness index and
	\item NEAT-TCP2 optimises only the overall throughput in a network.
	\end{itemize}	
	\item Evaluation of congestion control algorithms generated by NEAT-TCP and comparison with iTCP and TCP New Reno, by means of throughput, mean end-to-end delay, packet loss and fairness.
\end{itemize}
We present NEAT-TCP1, a neural network that is created by NEAT-TCP. NEAT-TCP1 achieves 69\% more fairness, 66\% less mean end-to-end delay and 71\% less packet loss in relation to TCP New Reno at the cost of 19\% less overall throughput.
\section{Outline}\label{sec:outline}
The following work is structured as follows: \autoref{ch:neatworking} describes how NEAT works, followed by related work in \autoref{ch:relatedwork}.  \autoref{ch:implementation} is solely about the technical implementation of this work. Thereafter the conducted experiments are described in \autoref{ch:experiments}, followed by their evaluation in \autoref{ch:evaluation}. Finally, in \autoref{ch:discussion}, the results and the work in general are discussed. Then this work is concluded in \autoref{ch:conclusion}.
